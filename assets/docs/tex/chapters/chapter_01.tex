\section{Problem sentence}

Facial emotions classification is a problem that continues improving as new technologies are developed. Existen distintas aproximaciones que seleccionan datasets sin criterio alguno de calidad el cual impacta en el entrenamiento y resultado del clasificador. Otro factor que impacta en la precisión es la configuración de la arquitectura (e.g. max-pooling, tamaños en kerneles de convolución). \textbf{Portabilidad, costo computacional, ¡tiempo de entrenamiento!}

\subsection{Portability}
\subsection{Computational cost}
\subsection{Reliability in datasets}
\subsection{Accuracy in classification models}

Con base en lo anterior, the problem lies in reliability and accuracy during classification, based on the architecture configuration and quality assurance of the dataset. Resumir problemática de costo comutacional \textbf{PARA} hacerlo portable.


\section{Motivation}

Responder:
\begin{itemize}
  \item ¿Cuáles son los beneficios de realizar la propuesta en un dispositivo portable?
  \item ¿A quiénes beneficiaría? (Ejemplos de enojo en autos $\Rightarrow$ accidentes.)
\end{itemize}


\section{Objectives}

Propose and test a neural network architecture against latest approaches to improve the accuracy classifying facial emotions with a reliable dataset.


\section{Research questions}

\begin{itemize}
  \item What are the advantages and drawbacks of latest NN/CNN architectures?
  \item Is it possible to propose a NN/CNN architecture with better accuracy for facial emotion classification?
  \item What are the datasets with the best settings?
  \item ¿Por qué un sistema de entrenamiento y clasificación debe ser portable?
\end{itemize}


\section{Summary}

\section{Thesis structure}
